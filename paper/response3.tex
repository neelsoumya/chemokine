\documentclass[10pt]{article}

% amsmath package, useful for mathematical formulas
\usepackage{amsmath}
% amssymb package, useful for mathematical symbols
\usepackage{amssymb}

% graphicx package, useful for including eps and pdf graphics
% include graphics with the command \includegraphics
\usepackage{graphicx}

% cite package, to clean up citations in the main text. Do not remove.
\usepackage{cite}

\usepackage{color} 
\usepackage[usenames,dvipsnames,table]{xcolor}


% Use doublespacing - comment out for single spacing
\usepackage{setspace} 
%\doublespacing

% Text layout
\topmargin 0.0cm
\oddsidemargin 0.5cm
\evensidemargin 0.5cm
\textwidth 16cm 
\textheight 21cm

% Bold the 'Figure #' in the caption and separate it with a period
% Captions will be left justified
\usepackage[labelfont=bf,labelsep=period,justification=raggedright]{caption}

% Use the PLoS provided bibtex style
%\bibliographystyle{plos2009}

% Remove brackets from numbering in List of References
\makeatletter
\renewcommand{\@biblabel}[1]{\quad#1.}
\makeatother


% Leave date blank
\date{}

\pagestyle{myheadings}
%% ** EDIT HERE **

\usepackage{multirow}

\usepackage{color}
\usepackage{csquotes}
\usepackage{ulem}

%% ** EDIT HERE **
%% PLEASE INCLUDE ALL MACROS BELOW

\definecolor{dkred}{rgb}{0.75,0,0}
\definecolor{dkgreen}{rgb}{0,0.5,0}
\definecolor{dkblue}{rgb}{0,0,0.75}
\definecolor{dkpurple}{rgb}{.375,0,.375}
\definecolor{gray}{rgb}{0.5,0.5,0.5}

\newcommand{\removed}[1]{{\color{dkred}\sout{#1}}}
%\newcommand{\removed}[1]{\textcolor{dkred}{#1}}
\newcommand{\new}[1]{{\color{dkgreen}#1}}

%\newenvironment{response}{\fontfamily{cms}\selectfont\small}{\par}
\newenvironment{response}{\fontfamily{cmr}}{\par}

\renewcommand{\rmdefault}{cmr}
\renewcommand{\sfdefault}{lmss}

% figure files reside in the figures/ directory
\graphicspath{
{figures/}
}

%% END MACROS SECTION

\begin{document}

% Title must be 150 characters or less
\begin{flushleft}
{\Large
\textbf{A spatial model of the efficiency of T cell search in the influenza-infected lung (Response to Reviewers)}
}
% Insert Author names, affiliations and corresponding author email.
\\
Drew Levin, 
Stephanie Forrest, 
Soumya Banerjee,
Candice Clay,
Judy Cannon, 
Melanie Moses, 
Frederick Koster
\end{flushleft}
\vspace{0.5cm}


%We thank the reviewers for their constructive and insightful comments.  We have done our best to address each point raised and feel the incorporation of the reviewers' comments has greatly improved the paper.  We include detailed responses to each comment below.

%We thank the reviewers for their efforts in providing insightful critiques. Below we have provided detailed responses to each individual point raised by the reviewers. As the response to Reviewer 2 is brief, we address Reviewer 2's critique first. We hope that the reviewers and editor will find these detailed responses sufficient to address all their current critiques and find our manuscript suitable for publication in the Journal of Theoretical Biology. 

We are grateful for the opportunity to (re)respond to Reviewer 1's
comments. We regret that Reviewer 1 was not satisfied by our
earlier responses, but we believe that we directly addressed all of
his/her concerns.  Here we provide additional evidence and discussion
to respond to Reviewer 1's comments.  We appreciate the time
spent by Reviewer 1 on the manuscript, but we respectfully disagree
with his/her opinion that we did not understand or address his/her
concerns.  At this point, we believe that we have adequately addressed
all comments from both Reviewer 1 and 2.

In summary, we agree with the statements and implications in Points 1,
2, 4 and 6. We have answered the questions in Points 3 and 5. Below we
include detailed responses to these points, but we respectfully submit
that nowhere in the review is there a reference to an incorrect
or incomplete analysis.

\section*{Reviewer 1}

%\textbf{Reviewer 1: The authors misinterpreted what my comments meant, and this is third time.}

\begin{enumerate}

\item \textbf{If you estimate chemokine production rate, I want to know how. You show no 
model fits, so I cannot trust your numbers/analysis. The fact that chemokine 
secretion rate does not impact clearance of pandemic virus raises a question -
why do you need to estimate secretion rate then?}

\begin{response}

This comment seems contradictory: On the one hand it asks for more
details about how we estimate chemokine production rate, and on the
other hand suggests that chemokine production rate doesn't need to
be estimated because it does not impact clearance.  Our analysis
agrees with his statement that it does not impact clearance. For
thoroughness, we explain how we estimated chemokine production rate
below.

There is no known literature with chemokine secretion rate
estimations.  In this paper, we present new empirical data of
\textit{in vitro} chemokine levels and extend a previously published
ODE model to estimate chemokine production rates.  We are as surprised
as Reviewer 1 that these production rates do not have a strong effect
on the agent-based modeling results (see sensitivity analyses).
Nevertheless, because our agent-based model examines chemokine
interactions, we require \textit{some} non-zero value for the production
rate.  We believe that fitting the extended ODE model to our new
empirical data is the best approach available to determine
biologically plausible values.  Our paper contains a subsection,
\textit{Models for Parameter Estimation}, that details the exact
process used to estimate the chemokine production rates (page 6).  We
fit the system of differential equations to empirical data using a
genetic algorithm to minimize the log squared error between the model
and the data (the genetic algorithm gave better fits than the more standard Levenberg-Marquardt nonlinear regression techniques).  

We have now added a figure displaying the model fits to the supplement (Fig. S1, also below) and Table S3 showing the root mean squared error of the fits for completeness.


%\begin{displayquote}
%``quote from paper"
%\end{displayquote}
%
%more response stuff

\end{response}

\item \textbf{Mitchell et al. work is on in vitro culture while you model in vivo lung 
infection. These are two completely different scenarios, parameter can and are 
likely to be different.}

\begin{response}

  We agree that in vitro and in vivo parameters may be different, but
  this is a caveat that applies throughout biological and biomedical
  sciences.  We, along with most other researchers in the field, use
  \textit{in vitro} generated values as a starting point when
  \textit{in vivo} values are not available, and we state when we do this explicitly
  in the manuscript.  In addition, the paper contains an extensive
  sensitivity analysis to address the concern that the values used to
  parameterize the model might vary from the initial starting
  values. We do not claim that the values we estimate correspond
  exactly to  \textit{in vivo} values. Instead, we use the best fit
  estimates that are currently available as a starting point, and
  conduct the sensitivity analysis to confirm the role of individual parameters on the model.
\end{response}

\item \textbf{You state that chemokine diffuses SLOWER that infection spread. What if it diffuses faster?}

\begin{response}

This question is answered directly in the paper---if diffusion of
chemokine is faster than infection spread, it does not impact our
results (Table 3, Figs. S4-S7).  In the paper, we argue that direct comparison of
viral and chemokine diffusion rates is not informative. 
Because T cells climb the chemokine gradient, a chemokine's effect is
determined by the location of its local maxima, while the viral effect
is determined by the spread and concentration of the virions
themselves.  We have edited the text to make this distinction more clear (\textit{Results: Spatial Effects}, page 10).

\begin{displayquote}
Finally, the spatial nature of our model reveals that \new{the locations of peak chemokine concentration}\removed{chemokines} can \new{move}\removed{diffuse} more slowly than the rate infected cells and virus expand, thus misdirecting the T cells.
\end{displayquote}

To complement the spatial analysis, we also performed multiple
sensitivity analyses of the model's free parameters.  This included scenarios where the chemokine's diffusion rate was increased by up to two orders of magnitude (Table 2).  Each sensitivity analysis showed that the chemokine diffusion rate was an insensitive parameter in the model (Table 3 and Supplement).

\end{response}


\item \textbf{Secretion of T cells is irrelevant parameter, recruitment to the focus of 
infection is the important one.}

\begin{response}
This comment restates as fact one of the conclusions of our analysis,
so we have no disagreement with this point.
%We agree with the reviewer that the T cell secretion rate is an
%insensitive parameter in our model as determined by our multiple
%sensitivity analyses.  
However, because our agent-based model investigates the recruitment of
T cells to the focus of infection, we required a non-zero secretion
rate to introduce T cells into the model.  To select a biologically
plausible value, we fit an ODE model to empirical data in our paper
(\textit{Models for Parameter Estimation}, page 7) and similarly to
the chemokine secretion rate discussed in Point 1.  We also agree that
the recruitment of T cells to the focus of the infection is important,
a point that is discussed in the \textit{Spatial Effects} subsection (page 9-10)
and in the discussion (page 15) as follows:

\begin{displayquote}
In our spatial model, CD8 T cells climb a chemokine gradient to find infected epithelial cells and
cluster at local maxima of chemokine concentration. Because T cells are clustered, they cannot cover the
expanding plaque effectively, where infected cells on the periphery become more highly dispersed as the
plaque grows. Thus, T cells in the model become redundant at a relatively low threshold, beyond which
additional T cells do not improve clearance rates.
\end{displayquote}

\end{response}

\item \textbf{You don't understand my point about virus control and relevance of your work
for interpretation of how CD8 T cells control flu replication.}

\begin{response}

The reviewer asks if changing certain parameters in our model would
change the outcome for virus control. Our model does not produce the
viral control that the reviewer hypothesizes it could produce, but
we believe that reporting results that do not conform to his/her
expectations should not be confused with ignoring the request to perform
the analysis.

The term `viral control' has been used by Reviewer 1 in three different contexts.  We have explicitly addressed each point in previous responses.  In summary:

\begin{enumerate}
\item \textit{You ignored my request to investigate if virus control can be good by T cells (irrespectively of the type of virus) if chemokines diffuse faster than virus particles. Please do the simulations, and rephrase your discussion depending on the results you find.}
\begin{displayquote}
We performed these simulations numerous times as part of the
sensitivity analyses, and Response 3 of this document reviews the
different roles of viral diffusion and chemokine diffusion.  
\end{displayquote}

\item \textit{One additional issue I thought of is the ability of T cells to recruit more T cells to the site of infection by also secreting cytokines or changing vasculature structure to allow more T cell entry. Will this improve viral control?}

\begin{displayquote}
Here we summarize our previous response (emphasis added). While no model can incorporate all potential aspects of T cell effects on the environment, we do address the reviewer’s question by varying the number of recruited T cells over a wide range (see Sensitivity Analysis: Table 3 and Figures S4-S7). If T cells secrete more cytokines or change the vasculature in order to recruit more T cells, increasing the number of T cells would take this potential effect into account without explicitly modeling cytokines or vasculature. As stated in the paper, we find that while T cells are sufficient to control seasonal H1N1, increasing T cell secretion from 1,257 per hour to 3,750 per hour \emph{does not affect} T cell control of pandemic flu (Figure S6).
\end{displayquote}

\item \textit{It seems to be that difference in pathogenesis of different viruses would come not from the inability of T cells to control virus spread but how much of the lung is being infected within 1-5 days post infection. Less virulent viruses infect fewer epithelian cells, are all cleared and this does not result in much pathology. Virulent viruses are able to infect much wider areas of the lung and T cells by clearing virus in the whole lung cause death - via immunopathology. Innate immunity is controlling early virus spread. Please investigate.}

\begin{displayquote}
In summary, our model does not attempt to include the entire immune response. It represents viral
spread using parameters that likely reflect innate control mechanisms (virus secretion, infectivity, diffusion
rate, and decay rate, see Tables 1 and 2). We appreciate the reviewer’s perspective, but our goal in this
paper is to study the effect of T cells and not to fully represent the interaction between influenza strains
and the entirety of the immune response (Introduction 3rd paragraph shown below). \\

\textit{``Specifically, we focus on the interactions between activated antigen-specific CD8 T cells, 
cytokines, and replicating influenza virus. ... Therefore, instead of developing a comprehensive
immune system model, we present a spatially explicit agent-based model (ABM) to describe T
cell interactions with chemotactic signals and a dynamically growing plaque." (page 1)} \\

We feel that careful investigation of the role of early innate control, with its many complex mechanisms, in
different influenza strains is best left to future studies.
\end{displayquote}

\end{enumerate}

\end{response}

\item \textbf{Please read the paper by Oreskes et al. Science 1994 about how mathematical 
modeling should be used to understand natural sciences.}


\begin{response}
We have reviewed Oreskes et. al.  and outline how our paper is
consistent with it.  This excellent paper observes that scientists often use terms incorrectly when describing their model's veracity which leads to incorrect claims and conclusions.  Specifically, Oreskes et. al. examines four terms: `verification', `validation', `calibration', and `confirmation'.
\begin{itemize}
\item \textbf{Verification:} Oreskes et. al. states that it is not
  possible to verify a numerical model by comparing it to an
  analytical model.  We use the term `verification' once (page 15, concluding paragraph) in the context of obtaining empirical data (emphasis added).
\begin{displayquote}
``Empirical \textbf{verification} of the model's sensitive parameters (viral response to IgM, infectivity, viral decay rate, viral diffusion rate, and viral production rate) will be valuable to future studies."
\end{displayquote}
\item \textbf{Validation:} This term does not appear in our paper.
\item \textbf{Calibration:} Oreskes et. al. warn of the dangers of
  using the calibration of free parameters to emprical data as a basis
  for claiming model validity.  We use the term `calibration' only
  once (Abstract) and make no claims that calibration implies model validity (emphasis added).
\begin{displayquote}
``We \textbf{calibrate} the model using viral and chemokine secretion rates we measure \textit{in vitro} together with values taken from literature."
\end{displayquote}
\item \textbf{Confirmation:} Oreskes et. al. states that observations consistent to a theory can `confirm' a theory, but warn that confirmation cannot lead to verification.  We use the term `confirm' twice and make no claims of verification or validity (emphasis added).

Page 11:
\begin{displayquote}
``The  five viral parameters are \textbf{confirmed} by the PRCC analysis to be significantly correlated with model output (Table 3)."
\end{displayquote}

Page 12:
\begin{displayquote}
``We \textbf{confirm} this effect in the model by first setting both relevant parameters (apoptosis time and T cell kill time) to zero, and as expected all three strains were cleared (Fig. 7)."
\end{displayquote}


\end{itemize}

We agree that models can inform and guide empirical research and use our model to propose
a hypothesis that additional T cells are limited in their
effectiveness due to the spatial effects of the chemokine gradient and
T cell chemotaxis.  We cannot find a place where our paper's
conclusions are inconsistent with the contents of Oreskes et al.
\end{response}

\end{enumerate}


\setcounter{figure}{0}
\renewcommand{\thefigure}{S\arabic{figure}}

\begin{figure}[b]
\begin{center}
\includegraphics[width=\columnwidth]{FigureS1.eps}
\caption{{\bf Preliminary Model 1 fits to data.}  Preliminary Model 1 (Eq. 1) was fit to experimental data (Table S1) using a genetic algorithm as described in the main paper.  sH1N1 IP-10 secretion exceeded measurement accuracy above 8500 pg/mL and these three values (empty red triangles) were not included in the model fitting.  RMSE values of the fits can be seen in Table S3.  IP-10 data and model fits shown in red (triangles), RANTES data and model fits shown in yellow (squares).}
\label{fig:fits}
\end{center}
\end{figure}


\end{document}









